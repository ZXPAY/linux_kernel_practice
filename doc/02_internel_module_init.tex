\section{Internel module init}

\textbf{Example code below}
\begin{lstlisting}[style=CStyle]
    #include <linux/kernel.h>
    #include <linux/module.h>
    
    MODULE_LICENSE("GPL");
    
    /*
    * Kernel module default initial function name
    */
    int init_module(void) {
        printk(KERN_INFO"%s: In init\n", __func__);
        return 0;
    }
    
    /*
    * Kernel module default exit function name
    */
    void cleanup_module(void) {
        printk(KERN_INFO"%s: In exit\n", __func__);
    }
    
    MODULE_AUTHOR("Danny Deng");
    MODULE_DESCRIPTION("Internel module example");     
\end{lstlisting}

\begin{itemize}
    \item First
    In my case the \textit{module\_init} define in this file \newline
    "/usr/src/linux-hwe-5.11-headers-5.11.0-41/include/linux/module.h"
    \item Second, the macro \textit{module\_init} will expand the function with "alias" to let function call the same thing with different name.
    \begin{lstlisting}[style=CStyle]
        /* Each module must use one module_init(). */
        #define module_init(initfn)					\
            static inline initcall_t __maybe_unused __inittest(void)		\
            { return initfn; }					\
            int init_module(void) __copy(initfn) __attribute__((alias(#initfn)));
        
        /* This is only required if you want to be unloadable. */
        #define module_exit(exitfn)					\
            static inline exitcall_t __maybe_unused __exittest(void)		\
            { return exitfn; }					\
            void cleanup_module(void) __copy(exitfn) __attribute__((alias(#exitfn)));
        
        #endif           
    \end{lstlisting}
\end{itemize}


